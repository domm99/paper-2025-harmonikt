\documentclass[conference]{IEEEtran}
\IEEEoverridecommandlockouts
% The preceding line is only needed to identify funding in the first footnote. If that is unneeded, please comment it out.
\usepackage{cite}
\usepackage{amsmath,amssymb,amsfonts}
\usepackage{algorithmic}
\usepackage{graphicx}
\usepackage{textcomp}
\usepackage{xcolor}
\usepackage{mytodonotes}
\usepackage{cleveref}
\def\BibTeX{{\rm B\kern-.05em{\sc i\kern-.025em b}\kern-.08em
    T\kern-.1667em\lower.7ex\hbox{E}\kern-.125emX}}
\begin{document}

\title{HarmoniKt: a Unifying Middleware for Heterogeneous Robot Swarms}


\author{
\IEEEauthorblockN{Manuel Andruccioli}
\IEEEauthorblockA{%\textit{dept. name of organization (of Aff.)} \\
\textit{University of Bologna}\\
Cesena, Italy \\
manuel.andruccioli@unibo.it}
\and
\IEEEauthorblockN{Angela Cortecchia}
\IEEEauthorblockA{%\textit{dept. name of organization (of Aff.)} \\
\textit{University of Bologna}\\
Cesena, Italy \\
angela.cortecchia@unibo.it}
\and
\IEEEauthorblockN{Davide Domini}
\IEEEauthorblockA{%\textit{dept. name of organization (of Aff.)} \\
\textit{University of Bologna}\\
Cesena, Italy \\
davide.domini@unibo.it}
\and
\IEEEauthorblockN{Nicolas Farabegoli}
\IEEEauthorblockA{%\textit{dept. name of organization (of Aff.)} \\
\textit{University of Bologna}\\
Cesena, Italy \\
nicolas.farabegoli@unibo.it}
\and
\IEEEauthorblockN{Silvia Mirri}
\IEEEauthorblockA{%\textit{dept. name of organization (of Aff.)} \\
\textit{University of Bologna}\\
Cesena, Italy \\
silvia.mirri@unibo.it}
\and
\IEEEauthorblockN{Danilo Pianini}
\IEEEauthorblockA{%\textit{dept. name of organization (of Aff.)} \\
\textit{University of Bologna}\\
Cesena, Italy \\
danilo.pianini@unibo.it}
\and
\IEEEauthorblockN{Riccardo Venanzi}
\IEEEauthorblockA{%\textit{dept. name of organization (of Aff.)} \\
\textit{University of Bologna}\\
Cesena, Italy \\
riccardo.venanzi@unibo.it}
\and
\IEEEauthorblockN{Mirko Viroli}
\IEEEauthorblockA{%\textit{dept. name of organization (of Aff.)} \\
\textit{University of Bologna}\\
Cesena, Italy \\
mirko.viroli@unibo.it}
}

% \author{
% \IEEEauthorblockN{
% Manuel Andruccioli,
% Angela Cortecchia,
% Davide Domini,
% Nicolas Farabegoli,\\
% Silvia Mirri,
% Riccardo Venanzi,
% Mirko Viroli
% }
% \IEEEauthorblockA{
% \textit{University of Bologna}\\
% Cesena, Italy \\
% \{manuel.andruccioli, angela.cortecchia, davide.domini, nicolas.farabegoli,\\
% silvia.mirri, riccardo.venanzi, mirko.viroli\}@unibo.it
% }
% }

\newcommand{\approach}{\textsc{HarmoniKt}}

\maketitle

\begin{abstract}
In recent years, 
 companies have increasingly invested in Industry 4.0 research 
 to automate repetitive tasks or activities that may be harmful to humans. 
% 
For instance, 
 large-scale warehouses such as those operated by Amazon rely heavily 
 on mobile robots to streamline logistics operations and ensure worker safety. 
% 
At the same time, 
 robot manufacturers are releasing more reliable platforms with advanced capabilities, 
 making them suitable for complex industrial tasks. 
% 
Despite this growing interest and technological progress, 
 significant challenges remain. 
% 
A key and non-trivial issue lies in the integration of heterogeneous robot fleets: 
 each vendor typically employs proprietary technologies and interfaces, 
 which hinders interoperability and limits the potential of multi-brand deployments.

In this work, 
 we introduce \approach{}, 
 an extensible middleware that addresses this challenge by introducing an abstraction layer 
 and providing a unified REST API for the control and management of heterogeneous robots. 
% 
Our solution has been validated in a physical industrial-like environment using a mixed fleet 
 of Boston Dynamics Spot and Mobile Industrial Robots (MiR). 
%
Furthermore, we present a comparative analysis showing that the access latency introduced 
 by our middleware is not significantly higher than that of direct robot access, 
 demonstrating the feasibility of unified robot management without compromising performance.
\end{abstract}

\begin{IEEEkeywords}
component, formatting, style, styling, insert
\end{IEEEkeywords}

\davide{ Page limit: 6}

\section{Introduction}\label{sec:intro}

\subsection{Context}
In the era of Industry 4.0, 
 the drive toward automation has intensified across manufacturing, 
 logistics, and service domains. 
% 
Many enterprises are investing heavily in research and deployment of robotic systems 
to take over repetitive tasks, reduce human exposure to hazardous environments, 
 and improve throughput and consistency. 
% 
Warehouse operations are a quintessential example: 
 major players like Amazon employ thousands of autonomous mobile robots to move inventory, 
 reduce picking times, and eliminate manual transport in large-scale facilities.
%
At the same time, the robotics industry has matured: 
 companies now produce platforms with increasing reliability, rich sensing suites, versatile locomotion, 
 and capability to execute complex behaviors (navigation, manipulation, perception). 
% 
The promise is that robotic fleets will become standard building blocks of the modern smart factory.

Yet, even with escalating investments and advancing robotics capabilities, 
 several challenges remain before these systems can fully deliver on their potential at scale.

\subsection{Research Gap}
\davide{
    if too long, remove some examples and put them in the related works
}
One of the major nontrivial obstacles is integration of heterogeneous robot fleets. 
%
In practice, 
 each manufacturer typically employs its own software stack, proprietary APIs, 
 communication protocols, and control paradigms. 
% 
This heterogeneity makes combining robots from multiple vendors into a seamless, 
 interoperable fleet extremely difficult, which in turn constrains system designers 
 to single-vendor lock-in or expensive custom wrappers. 
% 
In the literature of robot fleet management, 
 authors often note that much work focuses on task allocation, path planning, scheduling, 
 but relatively less tackles the problem of unified access and control across diverse platforms. 

Some middleware proposals have attempted to address interoperability in both industrial and academic contexts. 
%
For instance,
 TalkRoBots~\cite{ayaida2022fi} introduces a unified communication approach that bridges ROS, proprietary robot frameworks, 
 and IIoT devices in Industry 4.0 environments. 
% 
Zegarra et al.~\cite{cuadroszegarra2024jsan}, instead, propose an Internet of Robotic Things (IoRT) approach 
 to unify communication among different robot operating systems, focusing primarily on network-level interoperability 
 rather than providing unified control APIs. 
% 
The Pluggable Distributed Resource Allocator (PDRA)~\cite{rossi2020iros} presents a middleware layer that enables robots 
 to share computational tasks by intercepting executor requests and routing them across the fleet 
 to optimize energy consumption and latency. 
% 
While effective in distributed resource management, 
PDRA primarily addresses compute-resource sharing rather than unified control of heterogeneous robotic hardware. 
%
Similarly, in mission coordination contexts, architectures such as MissionControl~\cite{rodrigues2022jss} 
 support coalition formation among diverse robots, yet they still leave open the challenge of offering 
 a standardized control API across vendor-specific stacks.
%
Moreover, the trade-off between abstraction overhead and performance (e.g., latency, throughput) 
 is seldom thoroughly evaluated in real-world heterogeneous fleets.

\subsection{Contribution}
In this work, 
 we address precisely this gap by presenting \approach{}:
 an extensible middleware for unifying access to heterogeneous robot fleets. 
% 
\approach{} introduces an abstraction layer over diverse robot APIs and control paradigms, 
 exposing a common REST API for commanders or higher-level applications to issue motion commands, 
 query status, and orchestrate task execution, regardless of robot make or model. 
% 
Our architecture is modular and pluggable, 
 allowing future extensions to additional robot types and communication protocols with minimal effort.

We validate \approach{} in a physical testbed combining Boston Dynamics Spot robots 
 and MiR (Mobile Industrial Robots) platforms in an industrial-style environment. 
% 
We report on implementation challenges, integration strategies, and runtime behavior. 
%
Importantly, we include a comparative performance study: 
 we measure access latency through \approach{} and compare it to direct native access to each robot, 
 showing that the overhead introduced by our abstraction is limited and acceptable for practical deployments. 
% 
Thus, we demonstrate that unified robot management is feasible without significant performance compromise.

\subsection{Outline}
The rest of the paper is organized as follows:
 \Cref{sec:related} provides background and related works;
 \Cref{sec:arc} introduces \approach{} disscussing both the high level architecture and the implementation;
 \Cref{sec:eval} provides details on the real world evaluation;
 \Cref{sec:impact} discusses possible impacts of \approach{};
 and \Cref{sec:future} concludes the paper and outlines future research directions.

\section{Background and Related Works}\label{sec:related}

\subsection{Case Study Platforms: Boston Dynamics Spot and MiR}

\newpage

\section{\approach{} Middleware}\label{sec:arc}
\subsection{Architecture}

\subsection{Implementation}
\clearpage
.
\newpage

\section{Evaluation}\label{sec:eval}
\clearpage

\section{Opportunities and Challenges}\label{sec:impact}
\newpage
\section{Conclusions and future work}\label{sec:future}

\bibliographystyle{IEEEtran}
\bibliography{IEEEexample}

\end{document}
