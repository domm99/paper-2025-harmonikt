\documentclass[conference]{IEEEtran}
\IEEEoverridecommandlockouts
% The preceding line is only needed to identify funding in the first footnote. If that is unneeded, please comment it out.
\usepackage{cite}
\usepackage{amsmath,amssymb,amsfonts}
\usepackage{algorithmic}
\usepackage{graphicx}
\usepackage{textcomp}
\usepackage{xcolor}
\usepackage{mytodonotes}
\def\BibTeX{{\rm B\kern-.05em{\sc i\kern-.025em b}\kern-.08em
    T\kern-.1667em\lower.7ex\hbox{E}\kern-.125emX}}
\begin{document}

\title{HarmoniKt: a Unifying Middleware for Heterogeneous Robot Swarms}


\author{
\IEEEauthorblockN{Manuel Andruccioli}
\IEEEauthorblockA{%\textit{dept. name of organization (of Aff.)} \\
\textit{University of Bologna}\\
Cesena, Italy \\
manuel.andruccioli@unibo.it}
\and
\IEEEauthorblockN{Angela Cortecchia}
\IEEEauthorblockA{%\textit{dept. name of organization (of Aff.)} \\
\textit{University of Bologna}\\
Cesena, Italy \\
angela.cortecchia@unibo.it}
\and
\IEEEauthorblockN{Davide Domini}
\IEEEauthorblockA{%\textit{dept. name of organization (of Aff.)} \\
\textit{University of Bologna}\\
Cesena, Italy \\
davide.domini@unibo.it}
\and
\IEEEauthorblockN{Nicolas Farabegoli}
\IEEEauthorblockA{%\textit{dept. name of organization (of Aff.)} \\
\textit{University of Bologna}\\
Cesena, Italy \\
nicolas.farabegoli@unibo.it}
\and
\IEEEauthorblockN{Silvia Mirri}
\IEEEauthorblockA{%\textit{dept. name of organization (of Aff.)} \\
\textit{University of Bologna}\\
Cesena, Italy \\
silvia.mirri@unibo.it}
\and
\IEEEauthorblockN{Danilo Pianini}
\IEEEauthorblockA{%\textit{dept. name of organization (of Aff.)} \\
\textit{University of Bologna}\\
Cesena, Italy \\
danilo.pianini@unibo.it}
\and
\IEEEauthorblockN{Riccardo Venanzi}
\IEEEauthorblockA{%\textit{dept. name of organization (of Aff.)} \\
\textit{University of Bologna}\\
Cesena, Italy \\
riccardo.venanzi@unibo.it}
\and
\IEEEauthorblockN{Mirko Viroli}
\IEEEauthorblockA{%\textit{dept. name of organization (of Aff.)} \\
\textit{University of Bologna}\\
Cesena, Italy \\
mirko.viroli@unibo.it}
}

% \author{
% \IEEEauthorblockN{
% Manuel Andruccioli,
% Angela Cortecchia,
% Davide Domini,
% Nicolas Farabegoli,\\
% Silvia Mirri,
% Riccardo Venanzi,
% Mirko Viroli
% }
% \IEEEauthorblockA{
% \textit{University of Bologna}\\
% Cesena, Italy \\
% \{manuel.andruccioli, angela.cortecchia, davide.domini, nicolas.farabegoli,\\
% silvia.mirri, riccardo.venanzi, mirko.viroli\}@unibo.it
% }
% }


\maketitle

\begin{abstract}
In recent years, 
 companies have increasingly invested in Industry 4.0 research 
 to automate repetitive tasks or activities that may be harmful to humans. 
% 
For instance, 
 large-scale warehouses such as those operated by Amazon rely heavily 
 on mobile robots to streamline logistics operations and ensure worker safety. 
% 
At the same time, 
 robot manufacturers are releasing more reliable platforms with advanced capabilities, 
 making them suitable for complex industrial tasks. 
% 
Despite this growing interest and technological progress, 
 significant challenges remain. 
% 
A key and non-trivial issue lies in the integration of heterogeneous robot fleets: 
 each vendor typically employs proprietary technologies and interfaces, 
 which hinders interoperability and limits the potential of multi-brand deployments.

In this work, 
 we introduce HarmoniKt, 
 an extensible middleware that addresses this challenge by introducing an abstraction layer 
 and providing a unified REST API for the control and management of heterogeneous robots. 
% 
Our solution has been validated in a physical industrial-like environment using a mixed fleet 
 of Boston Dynamics Spot and Mobile Industrial Robots (MiR). 
%
Furthermore, we present a comparative analysis showing that the access latency introduced 
 by our middleware is not significantly higher than that of direct robot access, 
 demonstrating the feasibility of unified robot management without compromising performance.
\end{abstract}

\begin{IEEEkeywords}
component, formatting, style, styling, insert
\end{IEEEkeywords}

\davide{ Page limit: 6}

\section{Introduction}\label{sec:intro}
\clearpage

\section{Background and Related Works}\label{sec:related}
\newpage

\section{HarmoniKt Middleware}\label{sec:arc}
\subsection{Architecture}

\subsection{Implementation}
\clearpage
.
\newpage

\section{Evaluation}\label{sec:eval}
\clearpage

\section{Conclusions and future work}\label{sec:future}

\nocite{*}

\bibliographystyle{IEEEtran}
\bibliography{IEEEexample}

\end{document}
